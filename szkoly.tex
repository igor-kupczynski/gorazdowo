\documentclass[12pt,a4paper]{scrreprt}
\usepackage{hyperref}
\usepackage{polski}
\usepackage{epigraph}
\usepackage[utf8]{inputenc}

\setlength\epigraphwidth{12cm}
\setlength\epigraphrule{0pt}

\begin{document}
\title{Zbiorcza Szkoła Gminna im. Władysława Reymonta w Kołaczkowie}
\subtitle{\ldots{}ocalić od zapomnienia. Nr 1.}
\author{Aniceta Kubisiak}
\date{grudzień, 1980 rok}
\maketitle

\begin{abstract}
Motto --- 

\epigraph{\itshape My potrafimy karmić dusze milionów własnymi
  sercami, ale sami zawsze na próżno łakniemy odrobiny
  szczęścia.}{Władysław St.Reymont}

Oświata w Gminie Kołaczkowo do r.1980 (szkic monograficzny).

\newpage

``ocalić od zapomnienia''.

--- cytatem z ``Pieśni'' Gałczyńskiego nazwaliśmy naszą gazetkę
szkolną. Pragniemy utrwalić w pamięci mieszkańców Gminy i naszych
uczniów bardzo pobieżną historię szkół i nauczycieli - ludzi, którzy
swoje życie poświęcili dla kształtowania dusz ludzkich na tym
terenie. Dysponujemy bardzo skąpym materiałem faktograficznym
dotyczącym rozwoju oświaty i szkolnictwa. Postawą opracowania szkicu
monograficznego były kroniki szkół; wywiady przeprowadzone z
mieszkańcami poszczególnych wsi.

Kroniki szkół są niekompletne, a w niektórych placówkach uległy
zupełnemu zniszczeniu podczas okupacji. Być może, że w wykazach
nauczycieli w poszczególnych miejscowościach, są również pewne luki
(brak lub zniekształcenie nazwisk).

Jednak to, co zawarte jest w tej gazetce, pozwoli chociaż częściowo
zachować pamięć o tych miejscach i tych ludziach, którzy są najmilszym
wspomnieniem z czasów dzieciństwa i młodości każdego człowieka.

Aniceta Kubisiak
\end{abstract}

\chapter{Szkoła w Kołaczkowie}
Aniceta Kubisiak

\end{document}

%%% Local Variables: 
%%% mode: latex
%%% TeX-master: t
%%% End: 
